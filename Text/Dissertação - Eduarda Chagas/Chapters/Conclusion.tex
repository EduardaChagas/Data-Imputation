\chapter{Conclusions and Future Work}\label{chapter:conclusion}

The main objective of this work was the investigation of problems present in the methodology of symbolization of Bandt-Pompe and its applicability in the characterization of time series and images.
Interested in expanding the range of possible applications, we focus on investigating properties of transition graphs and their possible limitations present in the state of the art.
Another objective was the study of the joint distribution obtained by the descriptors of the Complexity-Entropy plane, as well as possible linear transformations in this space.
Thus, we have advanced in the state of the art by proposing some solutions to deal with scenarios not foreseen in the seminal article by Bandt-Pompe.

The work developed consists of the presentation of two new approaches and their respective applications in data analysis scenarios.
The first problem investigated was the characterization and classification of homogeneous patches of SAR image textures.
Knowing the limitations that exist in sequences that do not follow a unidimensional structure, the first step was the study of linearization methods, as well as the study of the different properties obtained in each class of data analyzed.
Thus, we propose a new approach for classification of remote sensing images, which consists of the following steps: linearization of data using Hilbert curves, generation of the weighted amplitude transition graphs for each sequence, extraction of information theory descriptors, and the classification of these features through the k-nearest neighbors algorithm.
As a result, we verified that we were able to obtain the same evaluation metrics of state-of-the-art handcraft algorithms with the use of only two features, which still provides the user with a new way of the general view of the problem, through the Complexity-Entropy plane.

The most important step of the application above was the proposal for a new generation of the ordinal patterns probability distribution, the weighted amplitude transition graph (WATG).
Considering that the magnitude variation of the targets' backscatter signal is an intrinsic characteristic of the analyzed region, we propose a modification in the ordinal patterns transition graph.
As can be seen in the literature, one of the limitations of traditional methods of ordinal patterns is the absence of amplitude information during the generation of the distribution.
In this way, we propose here the first variation of transition graphs that adds amplitude variation information to its edges.
An advantage of using WATG before classical techniques of adding amplitude information to the symbol histogram is in the higher level of information captured about the underlying dynamics of the system, even when using smaller values of patterns dimension.
Thus, we can obtain a greater degree of data discrimination more quickly and efficiently.

Another focus of the work was the development of an approach to build confidence regions in the Complexity-Entropy plane.
In order to improve the plane characterization capacity, we proposed the HC-PCA, which consists of calculating empirical regions in a new space generated by the linear transformation made by the principal component analysis algorithm.
We found that due to the correlation between the descriptors, proposals such as classical bivariate analysis, regressions and generalized linear models cannot describe the system dynamics well.
To verify the effectiveness of the proposal, we used these regions to capture the randomness of PRNGs in short sequences and thus characterize generators previously analyzed in the literature.
The technique proved to be fast, consistent and robust the addition of correlation structures.

This work presents several possibilities for future research.
For example, the use of WATG can be explored in different application scenarios.
Considering that its main characteristic consists of discriminating sequences with variations in amplitude along with the arrangement of its elements, its applicability is not restricted to remote sensing images.
In the context of SAR images, modifications can be made to increase the generalizability of the technique.
Its possible application in the segmentation of classes of regions is a challenging problem.

On the other hand, under the context of confidence regions, our work opens up a huge range of related research.
The study of regression models on correlated descriptors and the development of specific kernels for the Complexity-Entropy plane are fruitful possibilities for investigation.
We also emphasize the need for efforts to build representative metrics.
With the advancement of deep metric learning techniques~\citep{barros2020new}, we can explore the learning of projections in a linear transformation specific to the plane, which would allow progress to build specific machine learning algorithms for the Complexity-Entropy space.
This means that there is a lot of space to conduct further investigations regarding WATG and HC-PCA.