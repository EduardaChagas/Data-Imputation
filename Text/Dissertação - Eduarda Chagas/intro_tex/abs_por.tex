Nos últimos anos observamos um crescimento expressivo no número de aplicações inteligentes envolvendo análise, mineração e classificação de dados.
Com o aumento da complexidade das investigações a necessidade de abordagens simples, rápidas e com baixo custo computacional tornou-se fundamental.
No contexto de análise não-paramétrica de séries temporais, o uso da metodologia de simbolização de Bandt-Pompe tornou-se relevante.
Tendo como base o uso de padrões ordinais formados por meio dos elementos da série analisada, quando unido ao uso de descritores causais da teoria da informação mostrou-se apresentar um alto poder de caracterização da dinâmica geradora do processo subjacente aos dados.

Dentre os descritores, dois destes por apresentarem definições complementares vem recebendo um grande destaque na literatura: a entropia de Shannon, que neste contexto mensura o grau de desordem da distribuição dos padrões ordinais e a complexidade estatística, que por outro lado, representa o grau de dependência estrutural entre os elementos da sequência.
Em conjunto, tais features formam o plano Complexidade-Entropia, cujo o presente trabalho possui como objetivo evidenciar as suas principais lacunas, são elas:~(i) a ausência de métodos para construção de regiões de confiança e~(ii) a ambiguidade na formação dos símbolos provocada pela ausência de informações da amplitude de seus elementos.
Visando apresentar métodos alternativos para os problemas relatados, propomos duas soluções: uma modificação no grafo de transição de padrões ordinais, o Weighted Amplitude Transition Graph, que realiza o cálculo do peso de suas arestas usando informações de variação de amplitude entre os símbolos, e o HC-PCA, um método de geração de regiões de confiança empíricas sobre o plano.
Para validar nossas propostas, aplicações no contexto de sensoriamento remoto e análise de sequências de ruídos brancos foram desenvolvidas.

\textbf{Palavras-chave}: Simbolização de Bandt-Pompe, Padrões ordinais, Plano Complexidade-Entropia, Teoria da Informação.

